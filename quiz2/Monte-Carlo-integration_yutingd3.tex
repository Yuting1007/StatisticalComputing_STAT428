% Options for packages loaded elsewhere
\PassOptionsToPackage{unicode}{hyperref}
\PassOptionsToPackage{hyphens}{url}
%
\documentclass[
]{article}
\usepackage{lmodern}
\usepackage{amssymb,amsmath}
\usepackage{ifxetex,ifluatex}
\ifnum 0\ifxetex 1\fi\ifluatex 1\fi=0 % if pdftex
  \usepackage[T1]{fontenc}
  \usepackage[utf8]{inputenc}
  \usepackage{textcomp} % provide euro and other symbols
\else % if luatex or xetex
  \usepackage{unicode-math}
  \defaultfontfeatures{Scale=MatchLowercase}
  \defaultfontfeatures[\rmfamily]{Ligatures=TeX,Scale=1}
\fi
% Use upquote if available, for straight quotes in verbatim environments
\IfFileExists{upquote.sty}{\usepackage{upquote}}{}
\IfFileExists{microtype.sty}{% use microtype if available
  \usepackage[]{microtype}
  \UseMicrotypeSet[protrusion]{basicmath} % disable protrusion for tt fonts
}{}
\makeatletter
\@ifundefined{KOMAClassName}{% if non-KOMA class
  \IfFileExists{parskip.sty}{%
    \usepackage{parskip}
  }{% else
    \setlength{\parindent}{0pt}
    \setlength{\parskip}{6pt plus 2pt minus 1pt}}
}{% if KOMA class
  \KOMAoptions{parskip=half}}
\makeatother
\usepackage{xcolor}
\IfFileExists{xurl.sty}{\usepackage{xurl}}{} % add URL line breaks if available
\IfFileExists{bookmark.sty}{\usepackage{bookmark}}{\usepackage{hyperref}}
\hypersetup{
  pdftitle={Quiz2},
  pdfauthor={Yuting},
  hidelinks,
  pdfcreator={LaTeX via pandoc}}
\urlstyle{same} % disable monospaced font for URLs
\usepackage[margin=1in]{geometry}
\usepackage{color}
\usepackage{fancyvrb}
\newcommand{\VerbBar}{|}
\newcommand{\VERB}{\Verb[commandchars=\\\{\}]}
\DefineVerbatimEnvironment{Highlighting}{Verbatim}{commandchars=\\\{\}}
% Add ',fontsize=\small' for more characters per line
\usepackage{framed}
\definecolor{shadecolor}{RGB}{248,248,248}
\newenvironment{Shaded}{\begin{snugshade}}{\end{snugshade}}
\newcommand{\AlertTok}[1]{\textcolor[rgb]{0.94,0.16,0.16}{#1}}
\newcommand{\AnnotationTok}[1]{\textcolor[rgb]{0.56,0.35,0.01}{\textbf{\textit{#1}}}}
\newcommand{\AttributeTok}[1]{\textcolor[rgb]{0.77,0.63,0.00}{#1}}
\newcommand{\BaseNTok}[1]{\textcolor[rgb]{0.00,0.00,0.81}{#1}}
\newcommand{\BuiltInTok}[1]{#1}
\newcommand{\CharTok}[1]{\textcolor[rgb]{0.31,0.60,0.02}{#1}}
\newcommand{\CommentTok}[1]{\textcolor[rgb]{0.56,0.35,0.01}{\textit{#1}}}
\newcommand{\CommentVarTok}[1]{\textcolor[rgb]{0.56,0.35,0.01}{\textbf{\textit{#1}}}}
\newcommand{\ConstantTok}[1]{\textcolor[rgb]{0.00,0.00,0.00}{#1}}
\newcommand{\ControlFlowTok}[1]{\textcolor[rgb]{0.13,0.29,0.53}{\textbf{#1}}}
\newcommand{\DataTypeTok}[1]{\textcolor[rgb]{0.13,0.29,0.53}{#1}}
\newcommand{\DecValTok}[1]{\textcolor[rgb]{0.00,0.00,0.81}{#1}}
\newcommand{\DocumentationTok}[1]{\textcolor[rgb]{0.56,0.35,0.01}{\textbf{\textit{#1}}}}
\newcommand{\ErrorTok}[1]{\textcolor[rgb]{0.64,0.00,0.00}{\textbf{#1}}}
\newcommand{\ExtensionTok}[1]{#1}
\newcommand{\FloatTok}[1]{\textcolor[rgb]{0.00,0.00,0.81}{#1}}
\newcommand{\FunctionTok}[1]{\textcolor[rgb]{0.00,0.00,0.00}{#1}}
\newcommand{\ImportTok}[1]{#1}
\newcommand{\InformationTok}[1]{\textcolor[rgb]{0.56,0.35,0.01}{\textbf{\textit{#1}}}}
\newcommand{\KeywordTok}[1]{\textcolor[rgb]{0.13,0.29,0.53}{\textbf{#1}}}
\newcommand{\NormalTok}[1]{#1}
\newcommand{\OperatorTok}[1]{\textcolor[rgb]{0.81,0.36,0.00}{\textbf{#1}}}
\newcommand{\OtherTok}[1]{\textcolor[rgb]{0.56,0.35,0.01}{#1}}
\newcommand{\PreprocessorTok}[1]{\textcolor[rgb]{0.56,0.35,0.01}{\textit{#1}}}
\newcommand{\RegionMarkerTok}[1]{#1}
\newcommand{\SpecialCharTok}[1]{\textcolor[rgb]{0.00,0.00,0.00}{#1}}
\newcommand{\SpecialStringTok}[1]{\textcolor[rgb]{0.31,0.60,0.02}{#1}}
\newcommand{\StringTok}[1]{\textcolor[rgb]{0.31,0.60,0.02}{#1}}
\newcommand{\VariableTok}[1]{\textcolor[rgb]{0.00,0.00,0.00}{#1}}
\newcommand{\VerbatimStringTok}[1]{\textcolor[rgb]{0.31,0.60,0.02}{#1}}
\newcommand{\WarningTok}[1]{\textcolor[rgb]{0.56,0.35,0.01}{\textbf{\textit{#1}}}}
\usepackage{graphicx,grffile}
\makeatletter
\def\maxwidth{\ifdim\Gin@nat@width>\linewidth\linewidth\else\Gin@nat@width\fi}
\def\maxheight{\ifdim\Gin@nat@height>\textheight\textheight\else\Gin@nat@height\fi}
\makeatother
% Scale images if necessary, so that they will not overflow the page
% margins by default, and it is still possible to overwrite the defaults
% using explicit options in \includegraphics[width, height, ...]{}
\setkeys{Gin}{width=\maxwidth,height=\maxheight,keepaspectratio}
% Set default figure placement to htbp
\makeatletter
\def\fps@figure{htbp}
\makeatother
\setlength{\emergencystretch}{3em} % prevent overfull lines
\providecommand{\tightlist}{%
  \setlength{\itemsep}{0pt}\setlength{\parskip}{0pt}}
\setcounter{secnumdepth}{-\maxdimen} % remove section numbering
% https://github.com/rstudio/rmarkdown/issues/337
\let\rmarkdownfootnote\footnote%
\def\footnote{\protect\rmarkdownfootnote}

% https://github.com/rstudio/rmarkdown/pull/252
\usepackage{titling}
\setlength{\droptitle}{-2em}

\pretitle{\vspace{\droptitle}\centering\huge}
\posttitle{\par}

\preauthor{\centering\large\emph}
\postauthor{\par}

\predate{\centering\large\emph}
\postdate{\par}

\title{Quiz2}
\author{Yuting}
\date{2020/4/24}

\begin{document}
\maketitle

{
\setcounter{tocdepth}{2}
\tableofcontents
}
\hypertarget{topic-monte-carlo-integration}{%
\subsection{Topic: Monte-Carlo
integration}\label{topic-monte-carlo-integration}}

\hypertarget{problem-description}{%
\subsubsection{Problem description:}\label{problem-description}}

Write a function use the Monte-Carlo integration to get the integration
on {[}a, b{]} for arbitray function.

\(I(f)=\int_{a}^{b}(f(x))\), arbitrary f. Note: (a,b) is an arbitrary
interval, finite or infinite Assume finite.

\hypertarget{solution}{%
\subsubsection{Solution:}\label{solution}}

Use Monte-Carlo integration approach to estimate the integration,
standard error and 95\% CI.

\begin{Shaded}
\begin{Highlighting}[]
\CommentTok{#Function to be integrated over [a,b]}
\NormalTok{f <-}\StringTok{ }\ControlFlowTok{function}\NormalTok{(x)\{x\} }

\NormalTok{Integral <-}\StringTok{ }\ControlFlowTok{function}\NormalTok{(n,a,b,}\DataTypeTok{h=}\NormalTok{f)\{}
\NormalTok{u <-}\StringTok{ }\KeywordTok{runif}\NormalTok{(n,a,b) }
\NormalTok{Y <-}\StringTok{ }\NormalTok{(}\KeywordTok{h}\NormalTok{(u))}\OperatorTok{/}\NormalTok{(}\DecValTok{1}\OperatorTok{/}\NormalTok{(b}\OperatorTok{-}\NormalTok{a))}
\NormalTok{Int <-}\StringTok{ }\KeywordTok{mean}\NormalTok{(Y)}

\NormalTok{u <-}\StringTok{ }\KeywordTok{runif}\NormalTok{(n,a,b)}
\NormalTok{YY <-}\StringTok{ }\NormalTok{(}\KeywordTok{h}\NormalTok{(u)}\OperatorTok{/}\NormalTok{(}\DecValTok{1}\OperatorTok{/}\NormalTok{(b}\OperatorTok{-}\NormalTok{a)))}\OperatorTok{^}\DecValTok{2}
\NormalTok{SE <-}\StringTok{ }\KeywordTok{sqrt}\NormalTok{((}\KeywordTok{mean}\NormalTok{(YY)}\OperatorTok{-}\NormalTok{Int}\OperatorTok{^}\DecValTok{2}\NormalTok{)}\OperatorTok{/}\NormalTok{n)}
\NormalTok{CI <-}\StringTok{ }\KeywordTok{c}\NormalTok{(Int}\FloatTok{-1.96}\OperatorTok{*}\NormalTok{SE,Int}\FloatTok{+1.96}\OperatorTok{*}\NormalTok{SE)}
\KeywordTok{list}\NormalTok{(}\StringTok{"Int"}\NormalTok{=Int,}\StringTok{"SE"}\NormalTok{=SE, }\StringTok{"CI"}\NormalTok{=CI)\}}
\end{Highlighting}
\end{Shaded}

\hypertarget{evaluation}{%
\subsubsection{Evaluation:}\label{evaluation}}

Apply the function on both uniform distribution and exponential
distribution.

\begin{Shaded}
\begin{Highlighting}[]
\KeywordTok{Integral}\NormalTok{(}\DecValTok{1000}\NormalTok{,}\DecValTok{0}\NormalTok{,}\DecValTok{1}\NormalTok{,f)}
\end{Highlighting}
\end{Shaded}

\begin{verbatim}
## $Int
## [1] 0.4965
## 
## $SE
## [1] 0.009657
## 
## $CI
## [1] 0.4776 0.5154
\end{verbatim}

\begin{Shaded}
\begin{Highlighting}[]
\NormalTok{f1 <-}\StringTok{ }\ControlFlowTok{function}\NormalTok{(x)\{}\KeywordTok{exp}\NormalTok{(}\OperatorTok{-}\NormalTok{x)\}}
\KeywordTok{Integral}\NormalTok{(}\DecValTok{1000}\NormalTok{,}\DecValTok{0}\NormalTok{,}\DecValTok{10}\NormalTok{,f1)}
\end{Highlighting}
\end{Shaded}

\begin{verbatim}
## $Int
## [1] 1.153
## 
## $SE
## [1] 0.05212
## 
## $CI
## [1] 1.051 1.255
\end{verbatim}

\hypertarget{reference}{%
\subsubsection{Reference}\label{reference}}

\href{https://cran.r-project.org/web/packages/SI/vignettes/my-vignette.html}{Monte
Carlo Integration}

\end{document}
